\documentclass[a4paper]{article}

\usepackage[left=2cm,right=2cm,top=2cm,bottom=1.5cm]{geometry}

% General style
\usepackage{newunicodechar}
\newunicodechar{→}{\fontspec{Gentium Plus}→}
\newunicodechar{–}{--}
\newunicodechar{“}{``}
\newunicodechar{”}{''}
\newunicodechar{‘}{`}
\newunicodechar{’}{'}

% Formulas
\usepackage{amsmath}
\usepackage{amssymb}

% Internal references
\usepackage{hyperref}
\usepackage[capitalize]{cleveref}

% Figures
\usepackage{subcaption}
\usepackage[font=small,labelfont=it]{caption}
\usepackage{graphicx}
\usepackage{tikz}
\usetikzlibrary{arrows}
\usetikzlibrary{arrows.meta}
\usetikzlibrary{decorations.pathreplacing}
\usetikzlibrary{bayesnet}
\usepackage[linguistics,edges]{forest}
\usepackage{rotating}

% Bibliography
\usepackage[backend=biber,
            bibstyle=numeric-comp, %biblatex-sp-unified,
            citestyle=numeric-comp,
            sorting=none,
            maxcitenames=2,url=false,
            maxbibnames=10]{biblatex}
\addbibresource{library.bib}
\def\bibfont{\fontfamily{\rmdefault}\fontseries{m}\fontshape{n}\fontsize{9}{11}\selectfont}

\renewbibmacro*{doi+eprint+url}{%
  \printfield{doi}%
  \newunit\newblock%
  \iftoggle{bbx:eprint}{%
    \usebibmacro{eprint}%
  }{}%
  \newunit\newblock%
  \iffieldundef{doi}{%
    \usebibmacro{url+urldate}}%
  {}%
}

\newcommand{\glot}[2]{#1 {\scriptsize{[\texttt{\href{https://glottolog.org/resource/languoid/id/#2}{#2}}]}}}

% Code inclusion with syntax highlighting
\usepackage{minted}
\setminted{fontsize=\tiny,baselinestretch=0.9}



\title{Supplementary material: Clocks with bursts: Phylogenetic inference of schismogenesis in language evolution}
\date{}
\author{
  Gereon A. Kaiping$^{1}$,
  Nico Neureiter$^{1}$\\[2ex]
  $^{1}$Geographic Information Science Center, Universität Zürich, CH
}

\begin{document}
\maketitle
\renewcommand{\thepage}{S\arabic{page}} 
\renewcommand{\thesection}{S\arabic{section}}  
\renewcommand{\thetable}{S\arabic{table}}  
\renewcommand{\thefigure}{S\arabic{figure}}
\renewcommand{\figurename}{Figure} 

\section{Tree heights}
We have reconstructed phylogenetic trees for the Austronesian, Bantu, Indo-European and Sino-Tibetan language families using different clock models. Figure~\ref{fig:tree_height} shows the reconstructed tree height for each of these families according to strict and relaxed clock with and without bursts.

\begin{figure}[h]
  \centering
  \begin{subfigure}{0.4\textwidth}
    \includegraphics[width=\textwidth]{supplement/analysis/austronesian_treeheight.png}
    \caption{Austronesian}
    \label{fig:tree_height:austronesian}
  \end{subfigure}
  \begin{subfigure}{0.4\textwidth}
    \includegraphics[width=\textwidth]{supplement/analysis/bantu_treeheight.png}
    \caption{Bantu language}
    \label{fig:tree_height:bantu}
  \end{subfigure}

  \begin{subfigure}{0.4\textwidth}
    \includegraphics[width=\textwidth]{supplement/analysis/indoeuropean_treeheight.png}
    \caption{Indo-European}
    \label{fig:tree_height:indoeuropean}
  \end{subfigure}
  \begin{subfigure}{0.4\textwidth}
    \includegraphics[width=\textwidth]{supplement/analysis/sinotibetan_treeheight.png}
    \caption{Sino-Tibetan}
    \label{fig:tree_height:sinotibetan}
  \end{subfigure}
  
  \caption{The tree height, i.\,e. the age of the most recent common ancestor of all leaves, for the Austronesian, Bantu, Indo-European and Sino-Tibetan language families according to different clock models.}
  \label{fig:tree_height}
\end{figure}


\newpage
\section{Branch rate variation}

\begin{figure}[h]
  \centering
  \begin{subfigure}{0.4\textwidth}
    \includegraphics[width=\textwidth]{supplement/analysis/austronesian_clockrates.png}
    \caption{Austronesian}
    \label{fig:rate_variation:austronesian}
  \end{subfigure}
  \begin{subfigure}{0.4\textwidth}
    \includegraphics[width=\textwidth]{supplement/analysis/bantu_clockrates.png}
    \caption{Bantu}
    \label{fig:rate_variation:bantu}
  \end{subfigure}
  \begin{subfigure}{0.4\textwidth}
    \includegraphics[width=\textwidth]{supplement/analysis/indoeuropean_clockrates.png}
    \caption{Indo-European}
    \label{fig:rate_variation:indoeuropean}
  \end{subfigure}
  \begin{subfigure}{0.4\textwidth}
    \includegraphics[width=\textwidth]{supplement/analysis/sinotibetan_clockrates.png}
    \caption{Sino-Tibetan}
    \label{fig:rate_variation:sinotibetan}
  \end{subfigure}
  
  \caption{The branch rate variation in the four language families for the relaxed clock with and without bursts.}
  \label{fig:rate_variation}
\end{figure}

\end{document}